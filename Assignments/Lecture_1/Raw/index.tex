\documentclass{article}
\usepackage[paperheight=11in,paperwidth=8.5in, margin=0.5in]{geometry}
\usepackage{amssymb}
\usepackage{amsmath}

\begin{document}

\section*{Ecarma, Destin R. - 23100411}
\boldmath

\begin{enumerate}
	\item Recall the definition of a rational number, denoted as $\mathbb{Q}$. Prove that the Euler's number $e = \Sigma_{k=0}^\infty \frac{1}{k!} \notin \mathbb{Q}$. A factorial is defined as $k! = (k)(k-1)(k-2)(k-3)..., \forall k \in \mathbb{Z}^+$, note that $0! = 1$. Furthermore, a sum notation $\Sigma_{k=0}^\infty k = 0+ 1 + 2 + 3 +....+...$
	      \\~\\
	      Suppose $e$ is rational, such that $e = \frac{n}{d}$, where $n$ and $d$ are positive integers.

	      \begin{equation}
		      \frac{n}{d} = e = \sum_{k=0}^{\infty} \frac{1}{k!}
	      \end{equation}

	      Multiply both sides by $d!$, we get the following equation

	      \begin{equation}
		      \begin{split}
			      \frac{n}{d}d! & = d!\sum_{k=0}^{\infty} \frac{1}{k!}                                                                                                                                         \\
			      n(d - 1)!     & = d! \left[\frac{1}{0!} + \frac{1}{1!} + \frac{1}{2!} + \dots + \frac{1}{d!} + \frac{1}{(d+1)!} + \frac{1}{(d+2)!} + \frac{1}{(d+3)!} + \dots \right]                        \\
			                    & = \left(\frac{d!}{0!} + \frac{d!}{1!} + \frac{d!}{2!} + \dots + \frac{d!}{d!}\right) + \left[\frac{d!}{(d + 1)!}  + \frac{d!}{(d + 2)!} + \frac{d!}{(d + 3)!} + \dots\right] \\
			                    & = \sum_{k=0}^{d} \frac{d!}{k!} + \sum_{k=d+1}^{\infty} \frac{d!}{k!}
		      \end{split}
	      \end{equation}

	      The term $n(d - 1)!$ is clearly an integer. The first sum is also an integer, since $k \leq d$. The second sum we can simplify as follows

	      \begin{equation}
		      \frac{1}{d+1} + \frac{1}{(d+1)(d+2)} + \frac{1}{(d+1)(d+2)(d+3)} + \dots
	      \end{equation}

	      The second sum is clearly greater than $0$, and we can see (by considering respective terms) that

	      \begin{equation}
		      \left[\frac{1}{d+1} + \frac{1}{(d+1)(d+2)} + \frac{1}{(d+1)(d+2)(d+3)} + \dots\right] < \sum_{k=0}^{\infty} \left(\frac{1}{d+1}\right)^{(k+1)}
	      \end{equation}

	      The right-hand side of the inequality is a geometric series, and we used the formula to show that the second sum is going to be less than $1$

	      \begin{equation}
		      \begin{split}
			      \sum_{k=0}^{\infty} \left(\frac{1}{d+1}\right)^{(k+1)} & = \left[\frac{1}{d+1} + \frac{1}{(d+1)^2} + \frac{1}{(d+1)^3} + \dots\right] \\
			                                                             & = \frac{\frac{1}{d+1}}{1 - \frac{1}{d+1}} = \frac{1}{d} \leq 1
		      \end{split}
	      \end{equation}

	      With this logical statement, the second sum is greater than $0$ and less than $1$, which is not an integer. This is a contradiction, since $n(d-1)!$ is an integer, and the second sum is not an integer

	      \begin{equation}
		      n(d - 1)! \neq \sum_{k=0}^{d} \frac{d!}{k!} + \sum_{k=d+1}^{\infty} \frac{d!}{k!}
	      \end{equation}

	      Therefore, $e$ is irrational since $\mathbb{Z} = \mathbb{Z} + \overline{\mathbb{Z}}$ is simply impossible.



	\item Prove Minkowski's Inequality for sums, $\forall (p > 1, (a_k, b_k) > 0)$:
	      \setcounter{equation}{0}
	      \begin{equation}
		      \left(\sum_{k=1}^n |a_k + b_k|^p\right)^\frac{1}{p} \leq \left(\sum_{k=1}^n |a_k|^p\right)^\frac{1}{p} + \left(\sum_{k=1}^n |b_k|^p\right)^\frac{1}{p}
	      \end{equation}

	      To prove Minkowski's inequality, we can use Hölder's inequality.

	      (Hölder) if $p > 1$ and $q > 1$ are such that

	      \begin{equation}
		      \frac{1}{p} + \frac{1}{q} = 1
	      \end{equation}
	      then for all $x$ and $y \in \mathbb{R}^n$ we have
	      \begin{equation}
		      \sum_{k=1}^{n}|x_ky_k| \leq \left(\sum_{k=1}^{n}|x_k|^p\right)^\frac{1}{p}\left(\sum_{k=1}^{n}|y_k|^q\right)^\frac{1}{q}
	      \end{equation}

	      Proof of Minkowski Inequality. Observe that $(1)$ holds for the case $p = 1$ by summing both sides of the simple triangle inequality.

	      \begin{equation}
		      |a_k + b_k| \leq |a_k| + |b_k|
	      \end{equation}

	      We now concentrate on the case $p > 1$. For this case we show

	      \begin{equation}
		      \left(\sum_{k=1}^{n}(|a_k| + |b_k|)^p\right)^\frac{1}{p} \leq \left(\sum_{k=1}^{n}|a_k|^p\right)^\frac{1}{p} + \left(\sum_{k=1}^{n}|b_k|^p\right)^\frac{1}{p}
	      \end{equation}

	      so that the simple triangle inequality $(4)$ will imply $(1)$.
	      For all $a_k$ and $b_k$, we have

	      \begin{equation}
		      \begin{split}
			      (|a_k| + |b_k|)^p & = (|a_k| + |b_k|)(|a_k| + |b_k|)^{p-1}                    \\
			                        & = |a_k|(|a_k| + |b_k|)^{p-1} + |b_k|(|a_k| + |b_k|)^{p-1}
		      \end{split}
	      \end{equation}

	      Summing both sides of $(6)$ we obtain

	      \begin{equation}
		      \sum_{k=1}^{n}(|a_k| + |b_k|)^p = \sum_{k=1}^{n}|a_k|(|a_k| + |b_k|)^{p-1} + \sum_{k=1}^{n}|b_k|(|a_k| + |b_k|)^{p-1}
	      \end{equation}

	      Now, applying Hölder's inequality to both terms in the right-hand side of $(7)$, we have

	      \begin{align}
		      \sum_{k=1}^{n}|a_k|(|a_k| + |b_k|)^{p-1} & \leq \left(\sum_{k=1}^{n}|a_k|^p\right)^\frac{1}{p} \left(\sum_{k=1}^{n}[(|a_k| + |b_k|)^{p-1}]^q\right)^\frac{1}{q} \\
		      \sum_{k=1}^{n}|b_k|(|a_k| + |b_k|)^{p-1} & \leq \left(\sum_{k=1}^{n}|b_k|^p\right)^\frac{1}{p} \left(\sum_{k=1}^{n}[(|a_k| + |b_k|)^{p-1}]^q\right)^\frac{1}{q}
	      \end{align}

	      Thus, combining $(7)$ with $(8)$ and $(9)$, we obtain

	      \begin{equation}
		      \begin{split}
			      \sum_{k=1}^{n}(|a_k| + |b_k|)^p & \leq \left[\left(\sum_{k=1}^{n}|a_k|^p\right)^\frac{1}{p} + \left(\sum_{k=1}^{n}|b_k|^p\right)^\frac{1}{p}\right] \left(\sum_{k=1}^{n}(|a_k| + |b_k|)^{(p-1)q}\right)^\frac{1}{q} \\
			                                      & = \left[\left(\sum_{k=1}^{n}|a_k|^p\right)^\frac{1}{p} + \left(\sum_{k=1}^{n}|b_k|^p\right)^\frac{1}{p}\right] \left(\sum_{k=1}^{n}(|a_k| + |b_k|^p)\right)^\frac{1}{q}
		      \end{split}
	      \end{equation}

	      where we have used $(2)$. Rearanging $(10)$, we have

	      \begin{equation}
		      \frac{\sum_{k=1}^{n}(|a_k| + |b_k|)^p}{(\sum_{k=1}^{n}(|a_k| + |b_k|)^p)^\frac{1}{q}} \leq \left(\sum_{k=1}^{n}|a_k|^p\right)^\frac{1}{p} + \left(\sum_{k=1}^{n}|b_k|^p\right)^\frac{1}{p}
	      \end{equation}

	      Applying $(2)$ once more we have $(5)$ and so $(1)$ follows.


	\item Prove the triangle inequality $|x  +y| \leq |x| + |y|, \forall(x,y) \in \mathbb{R}$.
	      \\~\\
	      Case 1: Both $x$ and $y$ are nonnegative:

	      \begin{itemize}
		      \item If $x \geq 0$ and $y \geq 0$, then $|x + y| = x + y$ and $|x| + |y| = x + y$. Since $x + y \leq x + y$, the inequality holds.
	      \end{itemize}

	      Case 2: Both $x$ and $y$ are nonpositive:

	      \begin{itemize}
		      \item If $x \leq 0$ and $y \leq 0$, then $|x + y| = -(x + y)$ and $|x| + |y| = (-x) + (-y).$ Since $-(x + y) \leq (-x) + (-y)$, the inequality holds.
	      \end{itemize}

	      Case 3: One of $x$ or $y$ is nonnegative and the other is nonpositive:

	      \begin{itemize}
		      \item If $x \geq 0$ and $y \leq 0$, then $|x + y| = |x - (-y)|$ and $|x| + |y| = x + (-y)$. Since $|x - (-y)| \leq x + (-y)$, the inequality holds.
		      \item This method also applies to $x \leq 0$ and $y \geq 0$.
	      \end{itemize}

	      By examining all possible cases, we have shown that the triangle inequality holds $\forall(x, y) \in \mathbb{R}$. Therefore,the inequality $|x + y| \leq |x| + |y|$ is proven.

	\item Prove Sedrakyan's Lemma $\forall u_i, v_i \in \mathbb{R}^+$:
	      \setcounter{equation}{0}
	      \begin{equation}
		      \frac{(\sum_{i = 1}^n u_i)^2}{\sum_{i = 1}^n v_i} \leq \sum_{i=1}^n \frac{(u_i)^2}{v_i}
	      \end{equation}

	      To prove Sedrakyan's Lemma, we can use Cauchy-Schwarz Inequality.

	      Let's denote $x_i = \sqrt{vi}$ and $y_i = u_i$. Then the given inequality becomes

	      \begin{equation}
		      \left(\frac{\sum_{i=1}^{n}x_iy_i}{\sqrt{\sum_{i=1}^{n}v_i}}\right)^2 \leq \sum_{i=1}^{n}\frac{(x_i)^2 (y_i)^2}{v_i}
	      \end{equation}

	      This is essentially the squared form of the Cauchy-Schwarz Inequality, which states

	      \begin{equation}
		      \left(\sum_{i=1}^{n}a_i b_i\right)^2 \leq \left(\sum_{i=1}^{n}a_i^2\right) \left(\sum_{i=1}^{n}b_i^2\right)
	      \end{equation}

	      where

	      \begin{align}
		      a_i & = x_i\sqrt{\frac{v_i}{\sum_{i=1}^{n}v_i}}          \\
		      b_i & = \frac{y_i}{\sqrt{\frac{v_i}{\sum_{i=1}^{n}v_i}}}
	      \end{align}

	      By applying the Cauchy-Schwarz Inequality to $(4)$ and $(5)$, we get

	      \begin{equation}
		      \left(\sum_{i=1}^{n}x_i y_i \sqrt{\frac{v_i}{\sum_{i=1}^{n}v_i}}\right)^2 \leq \left(\sum_{i=1}^{n}(x_i)^2 \frac{v_i}{\sum_{i=1}^{n}v_i}\right) \left(\sum_{i=1}^{n}(y_i)^2 \frac{1}{\frac{v_i}{\sum_{i=1}^{n}v_i}}\right)
	      \end{equation}

	      Simplify

	      \begin{equation}
		      \left(\sum_{i=1}^{n}x_i y_i\right)^2 \leq \left(\sum_{i=1}^{n}(x_i)^2\right) \left(\sum_{i=1}^{n}(y_i)^2\right)
	      \end{equation}

	      Which is equivalent to

	      \begin{equation}
		      \left(\frac{\sum_{i=1}^{n}u_i}{\sqrt{\sum_{i=1}^{n}v_i}}\right)^2 \leq \sum_{i=1}^{n}\frac{(u_i)^2}{v_i}
	      \end{equation}

	      Hence, Sedrakyan's Lemma is proved.
\end{enumerate}
\end{document}