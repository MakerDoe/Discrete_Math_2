\documentclass{article}
\usepackage[paperheight=11in,paperwidth=8.5in, margin=0.5in]{geometry}
\usepackage{amssymb}
\usepackage{amsmath}
\usepackage{enumitem}

\begin{document}
\section*{Ecarma, Destin R. - 23100411}
\boldmath

\begin{enumerate}
	\item Consider the following experiment:

	      \begin{enumerate}
		      \item The experiment consists of $n$ identical trials.
		      \item The outcome of each trial falls into one of $k$ classes or cells.
		      \item The probability that the outcome of a single trial falls into cell $i$, is $p_i,i = 1,2,...,k$ and remains the same from trial to trial. Notice that $p_1+p_2+p_3+···+p_k =1.$
		      \item The trials are independent.
		      \item The random variables of interest are $Y_1 , Y_2 , . . . , Y_k$, where $Y_i$ equals the number of trials for which the outcome falls into cell $i$. Notice that $Y_1 +Y_2 +Y_3 +···+Y_k =n$.
		      \item The joint probability distribution $P(y) = \frac{n!}{y_1!y_2!y_3!...y_n!} P_1^{y_1} P_2^{y_2} P_3^{y_3}....P_n^{y_n}$
	      \end{enumerate}

	      \begin{enumerate}[label=1.\arabic*]
		      \item Show that the expected value $E[y] = \sum_{y} y \cdotp P(y) = n \cdot p_i$

		            The expected value of a random variable is calculated as the sum of all possible values of the variable weighted by their respective probabilities.

		            For the random variable $Y_i$, representing the number of trials where the outcome falls into cell $i$, the possible values range from $0$ to $n$.

		            \begin{equation}
			            E[Y_i] = \sum_{y_i=0}^{n} y_i \cdot P(Y_i = y_i)
		            \end{equation}

		            Given the joint probability distribution function $P(y)$, where $P(y) = \frac{n!}{y_1! y_2! y_3! ... y_n!}P_1^{y_1} P_2^{y_2} P_3^{y_3} ... P_n^{y_n}$, we can find $P(Y_i = y_i)$ by substituting $y_i$ into the expression fro $P(y)$.

		            \begin{align}
			            P(Y_i = y_i) & = \frac{n!}{y_i!(n - y_i)!}p_i^{y_i}(1 - p_i)^{n - y_i}                              \\
			            E[Y_i]       & = \sum_{y_i = 0}^{n} y_i \cdot \frac{n!}{y_i!(n - y_i)!}p_i^{y_i}(1 - p_i)^{n - y_i}
		            \end{align}

		            Expanding and simplifying this sum gives $E[Y_i] = np_i$.

		            Therefore, the expected value $E[Y] = \sum_{i=1}^{k} E[Y_i] = \sum_{i=1}^{k}np_i = n(p_1 + p_2 + p_3 + ... + p_k) = n$.

		      \item Given the following data

		            \begin{gather*}
			            \begin{bmatrix}
				            \begin{array}{c|c}
					            Age    & Proportion \\
					            18-24  & 0.18       \\
					            25-34  & 0.23       \\
					            35-44  & 0.16       \\
					            45-64  & 0.27       \\
					            65-100 & 0.16       \\
				            \end{array}
			            \end{bmatrix}
		            \end{gather*}

		            If $500$ adults are sampled  randomly, find the probability rhat the sample contains $100$ person between $18$ and $24$, $200$ between, $200$ between the ages of $25$ and $34$, and $200$ between the ages of $45$ and $64$. What is the expected value for to obtain a person in the $65$ and above?
		            \\~\\
		            We can calculate the probability using the given proportions for each age group.
		            \\~\\
		            Let $p_{18-24} = 0.18$, $p_{25-34} = 0.23$, $p_{35-44} = 0.16$, $p_{45-64} = 0.27$, and $p_{65-100} = 0.16$.
		            \\~\\
		            We want to find the probability of obtaining $100$ people between $18$ and $24$, $200$ people between $25$ and $34$, $200$ people between $45$ and $64$, and the remaining people aged $65$ and above.
		            The probability for this specific outcome is calculated using the multinomial probability formula

		            \begin{equation}
			            P = \frac{500!}{100! \cdot 200! \cdot 200! \cdot (500 - 100 - 200 - 200)!} \times (0.18)^{100} \times (0.23)^{200} \times (0.27)^{200} \times (0.16)^{500 - 100 - 200 - 200}
		            \end{equation}

		            You can compute this expression to find the probability.
		            \\~\\
		            To find the expected value for obtaining a person aged 65 and above, you would use the proportion for that age group, which is $p_{65-100} = 0.16$, and multiply it by the total number of samples $(500)$
		            \\~\\
		            Expected value for $65$ and above $ = 0.16 \times 500 = 80$.
	      \end{enumerate}
	\item Consider the following experiment
	      \setcounter{equation}{0}
	      \begin{enumerate}
		      \item The experiment consists of a fixed number, $n$, of identical trials.
		      \item Each trial results in one of two outcomes: success, $S$, or failure, $F$.
		      \item The probability of success on a single trial is equal to some value $p$ and remains the same from trial to trial. The probability of a failure is equal to $q = (1 - p)$.
		      \item The trials are independent.
		      \item The random variable of interest is $Y$, the number of successes observed during the $n$ trials.
	      \end{enumerate}

	      \begin{enumerate}[label=2.\arabic*]
		      \item From the following steps shown above, derive the probability distribution of the experiment.
		            \\~\\
		            The probability distribution of the experiment follows a binomial distribution, which can be represented as

		            \begin{equation}
			            P(Y_i = y_i) = (_{y_i}^{n}) p^{y_i} (1 - p)^{n - y_i}
		            \end{equation}

		            where

		            \begin{itemize}
			            \item $n$ is the number of trials,
			            \item $p$ is the probability of success on each trial,
			            \item $y_i$ is the number of successes observed during the trials, and
			            \item $(1 - p)$  is the probability of failure on each trial.
		            \end{itemize}
		      \item Show that the expectation of this probability distribution is $E[y] = \Sigma y \cdotp P(y) = np$.
		            \\~\\
		            The expectation of a binomial distribution is given by

		            \begin{equation}
			            E[Y_i] = \sum_{y_i=0}^{n} y_i \cdot P(Y_i = y_i) = np
		            \end{equation}

		      \item Experience has shown that $30\%$ of all persons afflicted by a certain illness recover. A drug company has developed a new medication. Ten people with the illness were selected at random and received the medication; nine recovered shortly thereafter. Suppose that the medication was absolutely worthless. What is the probability that at least nine of ten receiving the medication will recover?
		            \\~\\
		            Given that the medication is absolutely worthless, the probability of success $p$ is actually the same as the probability of recovery without the medication, which is $30\%$ or $0.3$. So, $p = 0.3$ and $q = 1 - p = 0.7$.

		            We want to find $P(Y \geq 9)$ when $n = 10$. We can calculate this using the binomial probability formula

		            \begin{equation}
			            \begin{split}
				            P(Y \geq 9) & = P(Y = 9) + P(Y = 10)                                                                               \\
				                        & = (_{9}^{10}) \times 0.3^{9} \times 0.7^{10 - 9} + (_{10}^{10}) \times 0.3^{10} \times 0.7^{10 - 10} \\
				                        & = 10 \times 0.3^{9} \times 0.7 + 1 \times 0.3^{10} \times 0.7^{0}                                    \\
				                        & = 10 \times 0.3^{9} \times 0.7 + 0.3^{10}                                                            \\
			            \end{split}
		            \end{equation}

		            So, the probability that at least nine out of ten individuals receiving the medication will recover is approximately $0.000144$.
		      \item Suppose that a lot of $5000$ electrical fuses contains $5\%$ defectives. If a sample of $5$ fuses is tested, find the probability of observing at least one defective.
		            \\~\\
		            Given that the lot contains $5\%$ defectives, $p = 0.05$ and $q = 1 - p = 0.95$. We are selecting a sample of $n = 5$ fuses.
		            \\~\\
		            We want to find $P(Y \geq 1)$. Using the complement rule, we find $P(Y \geq 1) = 1 - P(Y = 0)$.

		            \begin{equation}
			            P(Y \geq 1) = 1 - P(Y = 0) = 1 - (1 - 0.05)^5.
		            \end{equation}

		            So, the probability of observing at least one defective fuse in a sample of $5$ is approximately $0.2262$ or $22.62\%$.
	      \end{enumerate}

	\item Consider a probability distribution of $p(y) = \frac{\lambda^y}{y!} e^{-\lambda}$.
	      \setcounter{equation}{0}
	      \begin{enumerate}[label=3.\arabic*]
		      \item Find a general formula for the expected value of this distribution.
		            \\~\\
		            The expected value (or mean) of a probability distribution is given by the sum of each possible outcome multiplied by its probability. For this distribution, the possible outcomes are non-negative integers $(y = 0, 1, 2, ...)$. So, the expected value $E(y)$ is calculated as

		            \begin{equation}
			            E(y) = \sum_{y=0}^{\infty} y \cdot p(y)
		            \end{equation}

		            where $p(y) = \frac{\lambda^y}{y!}e^{-\lambda}$. Let's evaluate this

		            \begin{equation}
			            \begin{split}
				            E(y) & = \sum_{y=0}^{\infty} y \cdot \frac{\lambda^y}{y!}e^{-\lambda}                                                        \\
				                 & = \sum_{y=1}^{\infty} y \cdot \frac{\lambda^y}{y!}e^{-\lambda} \quad \text{since $y = 0$ yields 0}                    \\
				                 & = e^{-\lambda} \sum_{y=1}^{\infty}\frac{\lambda^y}{(y - 1)!}                                                          \\
				                 & = e^{-\lambda} \sum_{k=0}^{\infty}\frac{\lambda^{k+1}}{k!} \quad \text{where $k = y - 1$}                             \\
				                 & = e^{-\lambda} \lambda \sum_{k=0}^{\infty}\frac{\lambda^k}{k!}                                                        \\
				                 & = e^{-\lambda} \lambda e^{\lambda} = \lambda \quad \text{using $\sum_{k=0}^{\infty}\frac{\lambda^k}{k!} = e^\lambda$} \\
				                 & = \lambda
			            \end{split}
		            \end{equation}

		            So, the expected value of this distribution is simply $\lambda$.

		      \item A certain type of tree has seedlings randomly dispersed in a large area, with the mean density of seedlings being approximately five per square yard. If a forester randomly locates ten 1-square-yard sampling regions in the area, find the probability that none of the regions will contain seedlings.
		            \\~\\
		            Given that the mean density of seedlings is $5$ per square yard, the parameter $\lambda$ in our distribution is $5$. We want to find the probability that none of the ten 1-square-yard sampling regions contain seedlings. This is the same as finding the probability that the number of seedlings in each region is $0$.
		            \\~\\
		            Using the given probability distribution, with $\lambda = 5$. the probability of getting $0$ seedlings in one region is

		            \begin{equation}
			            p(y = 0) = \frac{5^0}{0!}e^{-5} = e^{-5}
		            \end{equation}

		            So, the probability that none of the regions contain seedlings is

		            \begin{equation}
			            P(\text{no seedlings in any region}) = (e^{-5})^10
		            \end{equation}

		            Therefore, the probability that none of the regions will contain seedlings is $e^{-50}$
	      \end{enumerate}
\end{enumerate}

\end{document}